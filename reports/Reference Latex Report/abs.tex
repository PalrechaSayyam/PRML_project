\begin{abstract}

Accurate molecular toxicity prediction is essential in early-stage drug discovery, yet existing models struggle with imbalanced datasets, noisy labels, and heterogeneous task difficulty. This work develops a unified benchmarking pipeline on the Tox21 dataset to evaluate baseline machine learning models, optimized ML classifiers, and graph-based deep learning models. Through the use of SMOTE balancing, Optuna hyperparameter tuning, and GraphConv/GAT architectures, the study observes complementary performance across different toxicity tasks.
To characterize these variations, the analysis computes INT-CHEM (intrinsic task difficulty) and EXT-CHEM (cross-task similarity using Optimal Transport Dataset Distance), revealing two distinct groups of tasks favoring either single-task models or knowledge-sharing approaches. Guided by these findings, the work introduces a hybrid model that concatenates GNN-derived graph embeddings with ECFP fingerprints and trains a unified Random Forest classifier. This hybrid representation consistently improves PR-AUC across tasks and outperforms both standalone GNNs and classical models. Overall, the study combines task-hardness analysis with model benchmarking to arrive at a simple, effective, and interpretable hybrid architecture for toxicity prediction.

%SegCaps 


%*** In this paper, we aimed to develop an automated image segmentation algorithm to detect and quantify the inflammation in the lung with the help of CNN and Capsule Network-based models. Convolution neural network(CNNs) has shown great potential over the last few years in medical image segmentation tasks. The new architecture which was introduced by Sabour et al. called Capsule Networks with Dynamic Routing has shown great potential results for digit recognition tasks and on small image classification tasks. The reason behind the success of Capsule Networks lies in the fact that they preserve more information about the input by replacement of max-pooling layers with convolutional strides and dynamic routing. The new proposed architecture named SegCaps introduced by Rodney LaLonde and Ulas Bagci expands the use of Capsule Networks for object segmentation task, we applied the proposed SegCaps to segment infectious region inside the lungs and further compared it with CNN based model named UNet++.
%Various ophthalmic procedures critically depend on high-quality images. For instance, efficiency of teleophthalmology, a framework to bring advanced eye care to remote regions, is determined by the capability of assessing diagnostic quality of ocular fundus photographs (FPs), and rejecting poor-quality ones at the source. In this context, we study algorithmic methods of classifying high- and low-quality FPs.  Crucially, diagnostic quality (DQ) -- determined by clinically, but not necessarily perceptually, significant structures -- is not synonymous with perceptual appeal. Yet, traditional methods handpick features individually (or in small subsets) to meet certain ad hoc perceptual requirements. In contrast, we investigate the efficacy of a comprehensive set of structure-preserving features, systematically generated by a deep scattering network (ScatNet). Specifically, we consider three advanced machine learning classifiers, train each using ScatNet as well as traditional features separately, and demonstrate that the former ensure significantly superior performance for each classifier under multiple criteria including classification accuracy.
\end{abstract}
    